%% Generated by Sphinx.
\def\sphinxdocclass{report}
\documentclass[a4paper,12pt,english]{sphinxmanual}
\ifdefined\pdfpxdimen
   \let\sphinxpxdimen\pdfpxdimen\else\newdimen\sphinxpxdimen
\fi \sphinxpxdimen=.75bp\relax
\ifdefined\pdfimageresolution
    \pdfimageresolution= \numexpr \dimexpr1in\relax/\sphinxpxdimen\relax
\fi
%% let collapsible pdf bookmarks panel have high depth per default
\PassOptionsToPackage{bookmarksdepth=5}{hyperref}
%% turn off hyperref patch of \index as sphinx.xdy xindy module takes care of
%% suitable \hyperpage mark-up, working around hyperref-xindy incompatibility
\PassOptionsToPackage{hyperindex=false}{hyperref}
%% memoir class requires extra handling
\makeatletter\@ifclassloaded{memoir}
{\ifdefined\memhyperindexfalse\memhyperindexfalse\fi}{}\makeatother

\PassOptionsToPackage{warn}{textcomp}

\catcode`^^^^00a0\active\protected\def^^^^00a0{\leavevmode\nobreak\ }
\usepackage{cmap}
\usepackage{xeCJK}
\usepackage{amsmath,amssymb,amstext}
\usepackage{babel}



\setmainfont{FreeSerif}[
  Extension      = .otf,
  UprightFont    = *,
  ItalicFont     = *Italic,
  BoldFont       = *Bold,
  BoldItalicFont = *BoldItalic
]
\setsansfont{FreeSans}[
  Extension      = .otf,
  UprightFont    = *,
  ItalicFont     = *Oblique,
  BoldFont       = *Bold,
  BoldItalicFont = *BoldOblique,
]
\setmonofont{FreeMono}[
  Extension      = .otf,
  UprightFont    = *,
  ItalicFont     = *Oblique,
  BoldFont       = *Bold,
  BoldItalicFont = *BoldOblique,
]



\usepackage[Bjornstrup]{fncychap}
\usepackage{sphinx}

\fvset{fontsize=\small,formatcom=\xeCJKVerbAddon}
\usepackage{geometry}


% Include hyperref last.
\usepackage{hyperref}
% Fix anchor placement for figures with captions.
\usepackage{hypcap}% it must be loaded after hyperref.
% Set up styles of URL: it should be placed after hyperref.
\urlstyle{same}


\usepackage{sphinxmessages}
\setcounter{tocdepth}{1}


    \usepackage{xeCJK}
    \usepackage{fontspec}
    \usepackage{indentfirst} % 中文首行缩进
	\let\cleardoublepage\clearpage
    \setlength{\parindent}{2em}
    \setCJKmainfont{Adobe Song Std}
    \setCJKmonofont[Scale=0.9]{Adobe Heiti Std}
    \setCJKfamilyfont{song}{Adobe Song Std}
    \setCJKfamilyfont{sf}{Adobe Song Std}
    \XeTeXlinebreaklocale "zh"          % 設定斷行演算法為中文
    \XeTeXlinebreakskip = 0pt plus 1pt  % 設定中文字距與英文字距
    

\title{dl\_server\_114 服务器用户手册}
\date{2021 年 12 月 08 日}
\release{}
\author{}
\newcommand{\sphinxlogo}{\vbox{}}
\renewcommand{\releasename}{}
\makeindex
\begin{document}

\ifdefined\shorthandoff
  \ifnum\catcode`\=\string=\active\shorthandoff{=}\fi
  \ifnum\catcode`\"=\active\shorthandoff{"}\fi
\fi

\pagestyle{empty}
\sphinxmaketitle
\pagestyle{plain}
\sphinxtableofcontents
\pagestyle{normal}
\phantomsection\label{\detokenize{index::doc}}


\sphinxAtStartPar
网页版访问公共文件夹:\sphinxhref{http://192.168.8.10:7890}{公共文件夹}

\sphinxAtStartPar
网页的登陆账户为:dl\_server\_114

\sphinxAtStartPar
公共文件夹的地址为:\sphinxurl{http://192.168.8.10:7890}。

\begin{sphinxadmonition}{caution}{小心:}
\sphinxAtStartPar
运算平台连接外网, \sphinxstylestrong{禁止} 运行涉密计算任务。
\end{sphinxadmonition}

\sphinxAtStartPar
账户均为普通账户,无 sudo 权限,仅可以访问自己的家目录,和公共文件夹 (\sphinxhref{../system/filesystem.html}{文件系统})。

\sphinxAtStartPar
服务器为公有资源,当使用 gpu 的时候,请按严格按照规则使用 slurm 调用系统资源。

\sphinxAtStartPar
可以通过 quickstart, 来快速上手。

\sphinxAtStartPar
快速链接
\begin{enumerate}
\sphinxsetlistlabels{\arabic}{enumi}{enumii}{}{.}%
\item {} 
\sphinxAtStartPar
{\hyperref[\detokenize{quickstart/index::doc}]{\sphinxcrossref{\DUrole{doc}{quickstart}}}}

\item {} 
\sphinxAtStartPar
{\hyperref[\detokenize{system/index::doc}]{\sphinxcrossref{\DUrole{doc}{系统}}}}

\item {} 
\sphinxAtStartPar
{\hyperref[\detokenize{accounts/index::doc}]{\sphinxcrossref{\DUrole{doc}{帐号}}}}

\item {} 
\sphinxAtStartPar
\sphinxhref{accounts/index.html\#id7}{密码}

\item {} 
\sphinxAtStartPar
{\hyperref[\detokenize{login/index::doc}]{\sphinxcrossref{\DUrole{doc}{登录}}}}

\item {} 
\sphinxAtStartPar
{\hyperref[\detokenize{job/index::doc}]{\sphinxcrossref{\DUrole{doc}{作业提交}}}}

\item {} 
\sphinxAtStartPar
{\hyperref[\detokenize{faq/index::doc}]{\sphinxcrossref{\DUrole{doc}{常见问题}}}}

\end{enumerate}


\chapter{quickstart}
\label{\detokenize{quickstart/index:quickstart}}\label{\detokenize{quickstart/index::doc}}
\sphinxAtStartPar
这里快速展示如何调用服务器资源。


\section{使用服务器四步骤}
\label{\detokenize{quickstart/index:id1}}
\sphinxAtStartPar
ssh 登陆服务器 \sphinxhyphen{}> 编写自己的脚本 \sphinxhyphen{}> 编写 slurm 脚本 \sphinxhyphen{}> 提交slurm 脚本


\section{资源如何选择?}
\label{\detokenize{quickstart/index:id2}}
\sphinxAtStartPar
可以先按下面的默认配置使用。

\sphinxAtStartPar
详情请见:\sphinxhref{../job/slurm.html}{Slurm 作业调度系统}


\section{连接服务器}
\label{\detokenize{quickstart/index:id3}}
\sphinxAtStartPar
目前服务器只允许通过 \sphinxhref{../login/ssh.html}{SSH 客户端}连接使用服务器。

\sphinxAtStartPar
了解服务器登陆方法,请查看 \sphinxhref{../login/index.html}{常见问题};


\section{提交 Hello world 单节点作业}
\label{\detokenize{quickstart/index:hello-world}}
\sphinxAtStartPar
以 Hello world 为例,演示服务器作业提交过程。
\begin{enumerate}
\sphinxsetlistlabels{\arabic}{enumi}{enumii}{}{.}%
\item {} 
\sphinxAtStartPar
撰写名为 hello\_world.py 代码如下

\end{enumerate}

\begin{sphinxVerbatim}[commandchars=\\\{\}]
\PYG{n+nb}{print}\PYG{p}{(}\PYG{l+s+s2}{\PYGZdq{}}\PYG{l+s+s2}{hello world!}\PYG{l+s+s2}{\PYGZdq{}}\PYG{p}{)}
\end{sphinxVerbatim}
\begin{enumerate}
\sphinxsetlistlabels{\arabic}{enumi}{enumii}{}{.}%
\setcounter{enumi}{1}
\item {} 
\sphinxAtStartPar
创建一个 anaconda 环境

\end{enumerate}
\begin{enumerate}
\sphinxsetlistlabels{\arabic}{enumi}{enumii}{}{.}%
\setcounter{enumi}{2}
\item {} 
\sphinxAtStartPar
编写一个名为 hello\_world.slurm 的作业脚本

\end{enumerate}

\begin{sphinxVerbatim}[commandchars=\\\{\}]
\PYG{c+ch}{\PYGZsh{}!/bin/bash}


\PYG{c+c1}{\PYGZsh{}SBATCH \PYGZhy{}\PYGZhy{}job\PYGZhy{}name=hello\PYGZus{}world}
\PYG{c+c1}{\PYGZsh{}SBATCH \PYGZhy{}\PYGZhy{}partition=debug  \PYGZsh{}\PYGZsh{} 这行在使用时请不要更改。}
\PYG{c+c1}{\PYGZsh{}SBATCH \PYGZhy{}\PYGZhy{}output=\PYGZpc{}j.out \PYGZsh{}\PYGZsh{} shell 返回的结果 将被输出到这个文档。}
\PYG{c+c1}{\PYGZsh{}SBATCH \PYGZhy{}\PYGZhy{}error=\PYGZpc{}j.err  \PYGZsh{}\PYGZsh{} shell 返回的错误 将被输出到这个文档。}
\PYG{c+c1}{\PYGZsh{}SBATCH \PYGZhy{}n 8 \PYGZsh{}\PYGZsh{} 作业使用服务器 cpu 的数量,一个任务最多调用 12 个cpu。超过 12 个可能导致系统卡死。}
\PYG{c+c1}{\PYGZsh{}SBATCH \PYGZhy{}\PYGZhy{}ntasks\PYGZhy{}per\PYGZhy{}node=8 \PYGZsh{}\PYGZsh{} 和上面设置相同即可。}

\PYG{n+nb}{ulimit} \PYGZhy{}l unlimited
\PYG{n+nb}{ulimit} \PYGZhy{}s unlimited

\PYG{c+c1}{\PYGZsh{} your codes}
\PYG{c+c1}{\PYGZsh{} 激活环境}
\PYG{n+nb}{source} activate py37
\PYG{c+c1}{\PYGZsh{} 运行自己的脚本}
python hello\PYGZus{}world.py
\end{sphinxVerbatim}
\begin{enumerate}
\sphinxsetlistlabels{\arabic}{enumi}{enumii}{}{.}%
\setcounter{enumi}{3}
\item {} 
\sphinxAtStartPar
提交到 SLURM

\end{enumerate}

\begin{sphinxVerbatim}[commandchars=\\\{\}]
\PYGZdl{} sbatch hello\PYGZus{}world.slurm
\end{sphinxVerbatim}
\begin{enumerate}
\sphinxsetlistlabels{\arabic}{enumi}{enumii}{}{.}%
\setcounter{enumi}{4}
\item {} 
\sphinxAtStartPar
查看已提交的作业

\end{enumerate}

\begin{sphinxVerbatim}[commandchars=\\\{\}]
\PYGZdl{} squeue
\end{sphinxVerbatim}

\noindent\sphinxincludegraphics{{12}.png}
\begin{enumerate}
\sphinxsetlistlabels{\arabic}{enumi}{enumii}{}{.}%
\setcounter{enumi}{5}
\item {} 
\sphinxAtStartPar
了解更多关于 slurm 作业系统 \sphinxhref{../job/slurm.html}{Slurm 作业调度系统}

\end{enumerate}


\chapter{系统}
\label{\detokenize{system/index:id1}}\label{\detokenize{system/index::doc}}

\section{服务器配置}
\label{\detokenize{system/index:id2}}
\sphinxAtStartPar
服务器采用 ubuntu 18.04 的操作系统,使用Linux自带的用户系统,配以 Slurm 作业调度系统,所有计算节点资源和存储资源,均可统一调用。

\sphinxAtStartPar
64 核心处理器

\sphinxAtStartPar
70GB RAM

\sphinxAtStartPar
8张 rtx 3090 24gb

\sphinxAtStartPar
服务器已安装 docker nvidia\sphinxhyphen{}docker2,普通用户可调用。


\subsection{文件系统}
\label{\detokenize{system/filesystem:id1}}\label{\detokenize{system/filesystem::doc}}
\begin{sphinxadmonition}{caution}{小心:}
\sphinxAtStartPar
注意:
\begin{enumerate}
\sphinxsetlistlabels{\arabic}{enumi}{enumii}{}{.}%
\item {} 
\sphinxAtStartPar
公共目录任何人都可操作,请勿在公共目录存放自己的重要数据。

\item {} 
\sphinxAtStartPar
ssd 共享目录 读取速度较快,但是容量小,故只允许存放公共数据集。

\item {} 
\sphinxAtStartPar
用户无法查看其他目录下的文件。

\end{enumerate}
\end{sphinxadmonition}

\noindent\sphinxincludegraphics{{1}.png}


\subsection{计算系统}
\label{\detokenize{system/computesystem:id1}}\label{\detokenize{system/computesystem::doc}}

\chapter{帐号}
\label{\detokenize{accounts/index:id1}}\label{\detokenize{accounts/index::doc}}

\section{帐号申请}
\label{\detokenize{accounts/index:id2}}
\sphinxAtStartPar
联系管理员申请账号


\section{密码}
\label{\detokenize{accounts/index:id3}}
\sphinxAtStartPar
默认的密码为:12345678


\subsection{密码自助更改}
\label{\detokenize{accounts/index:id4}}
\sphinxAtStartPar
如果您知道您的密码,并且需要更改密码,需要:
\begin{enumerate}
\sphinxsetlistlabels{\arabic}{enumi}{enumii}{}{.}%
\item {} 
\sphinxAtStartPar
连接到集群的登录节点

\item {} 
\sphinxAtStartPar
使用 \sphinxcode{\sphinxupquote{passwd}}指令

\item {} 
\sphinxAtStartPar
终端会提示您输入一次旧密码,并输入新密码

\end{enumerate}

\sphinxAtStartPar
至此,您已经更新了您的密码,请妥善保存新的密码。


\subsection{忘记密码}
\label{\detokenize{accounts/index:id5}}
\sphinxAtStartPar
请联系管理员


\chapter{登录}
\label{\detokenize{login/index:id1}}\label{\detokenize{login/index::doc}}
\sphinxAtStartPar
通过 SSH 登录


\bigskip\hrule\bigskip


\sphinxAtStartPar
本文将向大家介绍如何通过 SSH 远程登录到服务器。在阅读本文档之前,您需要具备 Linux/Unix、终端、MS\sphinxhyphen{}DOS、SSH
远程登录的相关知识,或者您可以阅读参考资料理解这些概念。

\sphinxAtStartPar
本文主要内容:
\begin{itemize}
\item {} 
\sphinxAtStartPar
使用 SSH 登录服务器注意事项;

\item {} 
\sphinxAtStartPar
首次登录准备,如信息采集、客户端下载、SSH 登录、SSH 文件传输、无密码登录等;

\item {} 
\sphinxAtStartPar
故障排除和反馈。

\end{itemize}

\sphinxAtStartPar
按照文档的操作说明将有助于您完成工作,谢谢您的配合!


\section{注意事项}
\label{\detokenize{login/index:id2}}\begin{itemize}
\item {} 
\sphinxAtStartPar
账号仅限于同一课题组的成员使用,请勿将账号借给他人使用。

\item {} 
\sphinxAtStartPar
恶意的 SSH 客户端软件会窃取您的密码,请在官网下载正版授权 SSH 客户端软件。

\end{itemize}


\section{准备}
\label{\detokenize{login/index:id3}}
\sphinxAtStartPar
通过 SSH 登录服务器,需要在客户端输入登录节点 IP 地址(或主机名),SSH 端口,SSH 用户名和密码。

\begin{sphinxVerbatim}[commandchars=\\\{\}]
\PYG{n}{SSH} \PYG{n}{login} \PYG{n}{node}\PYG{p}{:} \PYG{n}{服务器} \PYG{n}{ip} \PYG{n}{地址}
\PYG{n}{Username}\PYG{p}{:} \PYG{n}{YOUR\PYGZus{}USERNAME}
\PYG{n}{Password}\PYG{p}{:} \PYG{n}{YOUR\PYGZus{}PASSWORD}
\end{sphinxVerbatim}

\sphinxAtStartPar
登录节点 IP 地址为 192.168.8.10, SSH 端口为 22


\section{下载客户端}
\label{\detokenize{login/index:id4}}

\subsection{Windows}
\label{\detokenize{login/index:windows}}
\sphinxAtStartPar
Windows 推荐使用 Putty 免费客户端,下载后双击即可运行使用。可至 \sphinxhref{https://www.putty.org}{Putty 官网}下载。


\subsection{Linux/Unix/Mac}
\label{\detokenize{login/index:linux-unix-mac}}
\sphinxAtStartPar
Linux / Unix / Mac 操作系统拥有自己的 SSH 客户端,包括 ssh, scp, sftp 等。

\sphinxAtStartPar
SSH 登录


\bigskip\hrule\bigskip


\sphinxAtStartPar
下面介绍通过 SSH 登录


\subsection{Windows用户}
\label{\detokenize{login/index:id5}}
\sphinxAtStartPar
启动客户端 Putty,填写登录节点地址 login.hpc.sjtu.edu.cn,端口号采用默认值 22,然后点 Open 按钮,如下图所示:

\noindent\sphinxincludegraphics{{putty1}.png}

\sphinxAtStartPar
在终端窗口中,输入您的 SSH 用户名和密码进行登录:

\noindent\sphinxincludegraphics{{putty2}.png}

\sphinxAtStartPar
\sphinxstyleemphasis{提示:输入密码时,不显示字符,请照常进行操作,然后按回车键登录。}


\subsection{Linux/Unix/Mac用户使用SSH}
\label{\detokenize{login/index:linux-unix-macssh}}
\sphinxAtStartPar
Linux / Unix / Mac 用户可以使用终端中的命令行工具登录。下列语句指出了该节点的IP地址、用户名和SSH端口。

\begin{sphinxVerbatim}[commandchars=\\\{\}]
\PYGZdl{} ssh YOUR\PYGZus{}USERNAME@TARGET\PYGZus{}IP
\end{sphinxVerbatim}


\section{通过 SSH 传输文件}
\label{\detokenize{login/index:ssh}}
\sphinxAtStartPar
登录可以通过scp等方式将个人目录下的数据下载到本地,或者反向上传本地数据到个人目录。详情请参考具体请参考 {\hyperref[\detokenize{transport/index:label-transfer}]{\sphinxcrossref{\DUrole{std,std-ref}{数据传输}}}} 。

\sphinxAtStartPar
服务器网络资源有限,不推荐通过网络进行大批量的数据传输。


\chapter{数据传输}
\label{\detokenize{transport/index:label-transfer}}\label{\detokenize{transport/index:id1}}\label{\detokenize{transport/index::doc}}
\sphinxAtStartPar
服务器公网 IP 传输数据,理论速度上限约为 8MB/s,但是考虑到登录节点为大家共享使用,因此实际传输速度会偏低。对于数据传输,我们为您提供如下解决方案:
\begin{enumerate}
\sphinxsetlistlabels{\arabic}{enumi}{enumii}{}{.}%
\item {} 
\sphinxAtStartPar
少量数据传输,可以直接使用 SSH 客户端,或在本地使用 scp, rsync 命令向该节点发起传输请求。

\item {} 
\sphinxAtStartPar
少量数据,网页上传,登陆 wifi 后,打开浏览器输入 ip 地址 \sphinxurl{http://192.168.8.10:7890/}, 可以打开公共文件夹的网页文件管理器,页面登录后左上角有上下传选项。

\sphinxAtStartPar
何为公共目录,请查看 \sphinxhref{../system/filesystem.html}{文件系统}

\sphinxAtStartPar
账户名为:dl\_server\_114, 密码为:12345678

\noindent\sphinxincludegraphics{{11}.png}

\noindent\sphinxincludegraphics{{2}.png}

\item {} 
\sphinxAtStartPar
500GB以上数据传输,强烈建议您联系管理员,将硬盘等存储设备送至管理员传输。

\end{enumerate}


\chapter{作业提交}
\label{\detokenize{job/index:id1}}\label{\detokenize{job/index::doc}}
\sphinxAtStartPar
\sphinxhref{http://slurm.schedmd.com/}{SLURM} (Simple Linux Utility for Resource Management)是一种可扩展的工作负载管理器,已被全世界的国家超级计算机中心广泛采用。
它是免费且开源的,根据\sphinxhref{http://www.gnu.org/licenses/gpl.html}{GPL通用公共许可证}发行。

\sphinxAtStartPar
本文档将协助您通过 Slurm 管理作业。 在这里可以找到更多的工作样本。


\section{Slurm 概览}
\label{\detokenize{job/index:id2}}

\begin{savenotes}\sphinxattablestart
\centering
\begin{tabulary}{\linewidth}[t]{|T|T|}
\hline
\sphinxstyletheadfamily 
\sphinxAtStartPar
Slurm
&\sphinxstyletheadfamily 
\sphinxAtStartPar
功能
\\
\hline
\sphinxAtStartPar
sinfo
&
\sphinxAtStartPar
服务器状态
\\
\hline
\sphinxAtStartPar
squeue
&
\sphinxAtStartPar
排队作业状态
\\
\hline
\sphinxAtStartPar
sbatch
&
\sphinxAtStartPar
作业提交
\\
\hline
\sphinxAtStartPar
scontrol
&
\sphinxAtStartPar
查看和修改作业参数
\\
\hline
\sphinxAtStartPar
sacct
&
\sphinxAtStartPar
已完成作业报告
\\
\hline
\sphinxAtStartPar
scancel
&
\sphinxAtStartPar
删除作业
\\
\hline
\end{tabulary}
\par
\sphinxattableend\end{savenotes}


\section{\sphinxstyleliteralintitle{\sphinxupquote{sinfo}} 查看 服务器状态}
\label{\detokenize{job/index:sinfo}}

\begin{savenotes}\sphinxattablestart
\centering
\begin{tabulary}{\linewidth}[t]{|T|T|}
\hline
\sphinxstyletheadfamily 
\sphinxAtStartPar
Slurm
&\sphinxstyletheadfamily 
\sphinxAtStartPar
功能
\\
\hline
\sphinxAtStartPar
sinfo \sphinxhyphen{}N
&
\sphinxAtStartPar
查看节点级信息
\\
\hline
\sphinxAtStartPar
sinfo \sphinxhyphen{}N –states=idle
&
\sphinxAtStartPar
查看可用节点信息
\\
\hline
\sphinxAtStartPar
sinfo –partition=cpu
&
\sphinxAtStartPar
查看队列信息
\\
\hline
\sphinxAtStartPar
sinfo –help
&
\sphinxAtStartPar
查看所有选项
\\
\hline
\end{tabulary}
\par
\sphinxattableend\end{savenotes}

\sphinxAtStartPar
节点状态包括:

\sphinxAtStartPar
\sphinxcode{\sphinxupquote{drain}}(节点故障),\sphinxcode{\sphinxupquote{alloc}}(节点在用),\sphinxcode{\sphinxupquote{idle}}(节点可用),\sphinxcode{\sphinxupquote{down}}(节点下线),\sphinxcode{\sphinxupquote{mix}}(节点部分占用,但仍有剩余资源)。

\sphinxAtStartPar
查看总体资源信息:

\begin{sphinxVerbatim}[commandchars=\\\{\}]
\PYGZdl{} sinfo
PARTITION AVAIL  TIMELIMIT  NODES  STATE NODELIST
cpu         up  \PYG{l+m}{30}\PYGZhy{}00:00:0    \PYG{l+m}{656}   idle cas\PYG{o}{[}\PYG{l+m}{001}\PYGZhy{}656\PYG{o}{]}
dgx2        up  \PYG{l+m}{30}\PYGZhy{}00:00:0      \PYG{l+m}{8}   idle vol\PYG{o}{[}\PYG{l+m}{01}\PYGZhy{}08\PYG{o}{]}
\end{sphinxVerbatim}


\section{\sphinxstyleliteralintitle{\sphinxupquote{squeue}} 查看作业信息}
\label{\detokenize{job/index:squeue}}

\begin{savenotes}\sphinxattablestart
\centering
\begin{tabulary}{\linewidth}[t]{|T|T|}
\hline
\sphinxstyletheadfamily 
\sphinxAtStartPar
Slurm
&\sphinxstyletheadfamily 
\sphinxAtStartPar
功能
\\
\hline
\sphinxAtStartPar
squeue \sphinxhyphen{}j jobid
&
\sphinxAtStartPar
查看作业信息
\\
\hline
\sphinxAtStartPar
squeue \sphinxhyphen{}l
&
\sphinxAtStartPar
查看细节信息
\\
\hline
\sphinxAtStartPar
squeue \sphinxhyphen{}n HOST
&
\sphinxAtStartPar
查看特定节点作业信息
\\
\hline
\sphinxAtStartPar
squeue \sphinxhyphen{}A ACCOUNT\_LIST
&
\sphinxAtStartPar
查看ACCOUNT\_LIST的作业
\\
\hline
\sphinxAtStartPar
squeue
&
\sphinxAtStartPar
查看USER\_LIST的作业
\\
\hline
\sphinxAtStartPar
squeue –state=R
&
\sphinxAtStartPar
查看特定状态的作业
\\
\hline
\sphinxAtStartPar
squeue –start
&
\sphinxAtStartPar
查看排队作业的估计开始时间
\\
\hline
\sphinxAtStartPar
squeue –format=“LAYOUT”
&
\sphinxAtStartPar
使用给定的LAYOUT自定义输出
\\
\hline
\sphinxAtStartPar
squeue –help
&
\sphinxAtStartPar
查看所有的选项
\\
\hline
\end{tabulary}
\par
\sphinxattableend\end{savenotes}

\sphinxAtStartPar
作业状态包括\sphinxcode{\sphinxupquote{R}}(正在运行),\sphinxcode{\sphinxupquote{PD}}(正在排队),\sphinxcode{\sphinxupquote{CG}}(即将完成),\sphinxcode{\sphinxupquote{CD}}(已完成)。

\sphinxAtStartPar
默认情况下,\sphinxcode{\sphinxupquote{squeue}}只会展示在排队或在运行的作业。

\begin{sphinxVerbatim}[commandchars=\\\{\}]
\PYGZdl{} squeue
JOBID PARTITION     NAME     USER ST       TIME  NODES NODELIST\PYG{o}{(}REASON\PYG{o}{)}
\PYG{l+m}{18046}      dgx2   ZXLing     eenl  R    \PYG{l+m}{1}:35:53      \PYG{l+m}{1} vol04
\PYG{l+m}{17796}      dgx2   python    eexdl  R \PYG{l+m}{3}\PYGZhy{}00:22:04      \PYG{l+m}{1} vol02
\end{sphinxVerbatim}

\sphinxAtStartPar
显示您自己账户下的作业:

\begin{sphinxVerbatim}[commandchars=\\\{\}]
squeue
JOBID PARTITION     NAME     USER ST       TIME  NODES NODELIST\PYG{o}{(}REASON\PYG{o}{)}
\PYG{l+m}{17923}      dgx2     bash    hpcwj  R \PYG{l+m}{1}\PYGZhy{}12:59:05      \PYG{l+m}{1} vol05
\end{sphinxVerbatim}

\sphinxAtStartPar
\sphinxcode{\sphinxupquote{\sphinxhyphen{}l}}选项可以显示更细节的信息。

\begin{sphinxVerbatim}[commandchars=\\\{\}]
squeue
JOBID PARTITION     NAME     USER    STATE       TIME TIME\PYGZus{}LIMI  NODES NODELIST\PYG{o}{(}REASON\PYG{o}{)}
\PYG{l+m}{17923}      dgx2     bash    hpcwj  RUNNING \PYG{l+m}{1}\PYGZhy{}13:00:53 \PYG{l+m}{30}\PYGZhy{}00:00:00    \PYG{l+m}{1} vol05
\end{sphinxVerbatim}


\section{\sphinxstyleliteralintitle{\sphinxupquote{SBATCH}} 作业提交}
\label{\detokenize{job/index:sbatch}}
\sphinxAtStartPar
准备作业脚本然后通过\sphinxcode{\sphinxupquote{sbatch}}提交是 Slurm 的最常见用法。
为了将作业脚本提交给作业系统,Slurm 使用

\begin{sphinxVerbatim}[commandchars=\\\{\}]
\PYGZdl{} sbatch jobscript.slurm
\end{sphinxVerbatim}

\sphinxAtStartPar
Slurm 具有丰富的参数集。 以下最常用的。


\begin{savenotes}\sphinxattablestart
\centering
\begin{tabulary}{\linewidth}[t]{|T|T|}
\hline
\sphinxstyletheadfamily 
\sphinxAtStartPar
Slurm
&\sphinxstyletheadfamily 
\sphinxAtStartPar
含义
\\
\hline
\sphinxAtStartPar
\sphinxhyphen{}n {[}count{]}
&
\sphinxAtStartPar
总进程数
\\
\hline
\sphinxAtStartPar
–ntasks\sphinxhyphen{}per\sphinxhyphen{}node={[}count{]}
&
\sphinxAtStartPar
每台节点上的进程数
\\
\hline
\sphinxAtStartPar
\sphinxhyphen{}p {[}partition{]}
&
\sphinxAtStartPar
作业队列
\\
\hline
\sphinxAtStartPar
–job\sphinxhyphen{}name={[}name{]}
&
\sphinxAtStartPar
作业名
\\
\hline
\sphinxAtStartPar
–output={[}file\_name{]}
&
\sphinxAtStartPar
标准输出文件
\\
\hline
\sphinxAtStartPar
–error={[}file\_name{]}
&
\sphinxAtStartPar
标准错误文件
\\
\hline
\sphinxAtStartPar
–time={[}dd\sphinxhyphen{}hh:mm:ss{]}
&
\sphinxAtStartPar
作业最大运行时长
\\
\hline
\sphinxAtStartPar
–exclusive
&
\sphinxAtStartPar
独占节点
\\
\hline
\sphinxAtStartPar
–mail\sphinxhyphen{}type={[}type{]}
&
\sphinxAtStartPar
通知类型,可选 all, fail,
end,分别对应全通知、故障通知、结束通知
\\
\hline
\sphinxAtStartPar
–mail\sphinxhyphen{}user={[}mail\_address{]}
&
\sphinxAtStartPar
通知邮箱
\\
\hline
\sphinxAtStartPar
–nodelist={[}nodes{]}
&
\sphinxAtStartPar
偏好的作业节点
\\
\hline
\sphinxAtStartPar
–exclude={[}nodes{]}
&
\sphinxAtStartPar
避免的作业节点
\\
\hline
\sphinxAtStartPar
–depend={[}state:job\_id{]}
&
\sphinxAtStartPar
作业依赖
\\
\hline
\sphinxAtStartPar
–array={[}array\_spec{]}
&
\sphinxAtStartPar
序列作业
\\
\hline
\end{tabulary}
\par
\sphinxattableend\end{savenotes}

\sphinxAtStartPar
这是一个名为\sphinxcode{\sphinxupquote{cpu.slurm}}的作业脚本,该脚本向cpu队列申请1个节点40核,并在作业完成时通知。在此作业中执行的命令是\sphinxcode{\sphinxupquote{/bin/hostname}}。

\begin{sphinxVerbatim}[commandchars=\\\{\}]
\PYG{c+ch}{\PYGZsh{}!/bin/bash}

\PYG{c+c1}{\PYGZsh{}SBATCH \PYGZhy{}\PYGZhy{}job\PYGZhy{}name=hostname}
\PYG{c+c1}{\PYGZsh{}SBATCH \PYGZhy{}\PYGZhy{}partition=cpu}
\PYG{c+c1}{\PYGZsh{}SBATCH \PYGZhy{}N 1}
\PYG{c+c1}{\PYGZsh{}SBATCH \PYGZhy{}\PYGZhy{}mail\PYGZhy{}type=end}
\PYG{c+c1}{\PYGZsh{}SBATCH \PYGZhy{}\PYGZhy{}mail\PYGZhy{}user=YOU@EMAIL.COM}
\PYG{c+c1}{\PYGZsh{}SBATCH \PYGZhy{}\PYGZhy{}output=\PYGZpc{}j.out}
\PYG{c+c1}{\PYGZsh{}SBATCH \PYGZhy{}\PYGZhy{}error=\PYGZpc{}j.err}

/bin/hostname
\end{sphinxVerbatim}

\sphinxAtStartPar
用以下方式提交作业:

\begin{sphinxVerbatim}[commandchars=\\\{\}]
sbatch cpu.slurm
\end{sphinxVerbatim}

\sphinxAtStartPar
\sphinxcode{\sphinxupquote{squeue}}可用于检查作业状态。用户可以在作业执行期间通过SSH登录到计算节点。输出将实时更新到文件{[}jobid{]}
.out和{[}jobid{]} .err。

\sphinxAtStartPar
这里展示一个更复杂的作业要求,其中将启动80个进程,每台主机40个进程。

\begin{sphinxVerbatim}[commandchars=\\\{\}]
\PYG{c+ch}{\PYGZsh{}!/bin/bash}

\PYG{c+c1}{\PYGZsh{}SBATCH \PYGZhy{}\PYGZhy{}job\PYGZhy{}name=LINPACK}
\PYG{c+c1}{\PYGZsh{}SBATCH \PYGZhy{}\PYGZhy{}partition=cpu}
\PYG{c+c1}{\PYGZsh{}SBATCH \PYGZhy{}n 80}
\PYG{c+c1}{\PYGZsh{}SBATCH \PYGZhy{}\PYGZhy{}ntasks\PYGZhy{}per\PYGZhy{}node=40}
\PYG{c+c1}{\PYGZsh{}SBATCH \PYGZhy{}\PYGZhy{}mail\PYGZhy{}type=end}
\PYG{c+c1}{\PYGZsh{}SBATCH \PYGZhy{}\PYGZhy{}mail\PYGZhy{}user=YOU@EMAIL.COM}
\PYG{c+c1}{\PYGZsh{}SBATCH \PYGZhy{}\PYGZhy{}output=\PYGZpc{}j.out}
\PYG{c+c1}{\PYGZsh{}SBATCH \PYGZhy{}\PYGZhy{}error=\PYGZpc{}j.err}

\PYG{c+c1}{\PYGZsh{} your codes}
...
\end{sphinxVerbatim}

\sphinxAtStartPar
以下作业请求4张GPU卡,其中1个CPU进程管理1张GPU卡。

\begin{sphinxVerbatim}[commandchars=\\\{\}]
\PYG{c+ch}{\PYGZsh{}!/bin/bash}

\PYG{c+c1}{\PYGZsh{}SBATCH \PYGZhy{}\PYGZhy{}job\PYGZhy{}name=GPU\PYGZus{}HPL}
\PYG{c+c1}{\PYGZsh{}SBATCH \PYGZhy{}\PYGZhy{}partition=dgx2}
\PYG{c+c1}{\PYGZsh{}SBATCH \PYGZhy{}n 4}
\PYG{c+c1}{\PYGZsh{}SBATCH \PYGZhy{}\PYGZhy{}ntasks\PYGZhy{}per\PYGZhy{}node=4}
\PYG{c+c1}{\PYGZsh{}SBATCH \PYGZhy{}\PYGZhy{}gres=gpu:4}
\PYG{c+c1}{\PYGZsh{}SBATCH \PYGZhy{}\PYGZhy{}mail\PYGZhy{}type=end}
\PYG{c+c1}{\PYGZsh{}SBATCH \PYGZhy{}\PYGZhy{}mail\PYGZhy{}user=YOU@MAIL.COM}
\PYG{c+c1}{\PYGZsh{}SBATCH \PYGZhy{}\PYGZhy{}output=\PYGZpc{}j.out}
\PYG{c+c1}{\PYGZsh{}SBATCH \PYGZhy{}\PYGZhy{}error=\PYGZpc{}j.err}

\PYG{c+c1}{\PYGZsh{} your codes}
...
\end{sphinxVerbatim}

\sphinxAtStartPar
以下作业启动一个3任务序列(从0到2),每个任务需要1个CPU内核。关于 π 集群上的Python,您可以查阅我们的\sphinxhref{https://docs.hpc.sjtu.edu.cn/application/Python/}{Python文档}。

\begin{sphinxVerbatim}[commandchars=\\\{\}]
\PYG{c+ch}{\PYGZsh{}!/bin/bash}

\PYG{c+c1}{\PYGZsh{}SBATCH \PYGZhy{}\PYGZhy{}job\PYGZhy{}name=python\PYGZus{}array}
\PYG{c+c1}{\PYGZsh{}SBATCH \PYGZhy{}\PYGZhy{}mail\PYGZhy{}user=YOU@MAIL.COM}
\PYG{c+c1}{\PYGZsh{}SBATCH \PYGZhy{}\PYGZhy{}mail\PYGZhy{}type=ALL}
\PYG{c+c1}{\PYGZsh{}SBATCH \PYGZhy{}\PYGZhy{}ntasks=1}
\PYG{c+c1}{\PYGZsh{}SBATCH \PYGZhy{}\PYGZhy{}time=00:30:00}
\PYG{c+c1}{\PYGZsh{}SBATCH \PYGZhy{}\PYGZhy{}array=0\PYGZhy{}2}
\PYG{c+c1}{\PYGZsh{}SBATCH \PYGZhy{}\PYGZhy{}output=python\PYGZus{}array\PYGZus{}\PYGZpc{}A\PYGZus{}\PYGZpc{}a.out}
\PYG{c+c1}{\PYGZsh{}SBATCH \PYGZhy{}\PYGZhy{}output=python\PYGZus{}array\PYGZus{}\PYGZpc{}A\PYGZus{}\PYGZpc{}a.err}

module load miniconda2/4.6.14\PYGZhy{}gcc\PYGZhy{}4.8.5

\PYG{n+nb}{source} activate YOUR\PYGZus{}ENV\PYGZus{}NAME

\PYG{n+nb}{echo} \PYG{l+s+s2}{\PYGZdq{}SLURM\PYGZus{}JOBID: \PYGZdq{}} \PYG{n+nv}{\PYGZdl{}SLURM\PYGZus{}JOBID}
\PYG{n+nb}{echo} \PYG{l+s+s2}{\PYGZdq{}SLURM\PYGZus{}ARRAY\PYGZus{}TASK\PYGZus{}ID: \PYGZdq{}} \PYG{n+nv}{\PYGZdl{}SLURM\PYGZus{}ARRAY\PYGZus{}TASK\PYGZus{}ID}
\PYG{n+nb}{echo} \PYG{l+s+s2}{\PYGZdq{}SLURM\PYGZus{}ARRAY\PYGZus{}JOB\PYGZus{}ID: \PYGZdq{}} \PYG{n+nv}{\PYGZdl{}SLURM\PYGZus{}ARRAY\PYGZus{}JOB\PYGZus{}ID}

python \PYGZlt{} vec\PYGZus{}\PYG{l+s+si}{\PYGZdl{}\PYGZob{}}\PYG{n+nv}{SLURM\PYGZus{}ARRAY\PYGZus{}TASK\PYGZus{}ID}\PYG{l+s+si}{\PYGZcb{}}.py
\end{sphinxVerbatim}


\section{参考资料}
\label{\detokenize{job/index:id3}}\begin{itemize}
\item {} 
\sphinxAtStartPar
\sphinxhref{http://slurm.schedmd.com}{SLURM Workload Manager}

\item {} 
\sphinxAtStartPar
\sphinxhref{http://www.accre.vanderbilt.edu/?page\_id=2154}{ACCRE’s SLURM
Documentation}

\item {} 
\sphinxAtStartPar
\sphinxhref{http://www.nccs.nasa.gov/images/intro-to-slurm-20131218.pdf}{Introduction to SLURM (NCCS lunchtime
series)}

\end{itemize}


\chapter{常见问题}
\label{\detokenize{faq/index:id1}}\label{\detokenize{faq/index::doc}}

\chapter{文档下载}
\label{\detokenize{download/index:id1}}\label{\detokenize{download/index::doc}}
\sphinxAtStartPar
\sphinxhref{/\_static/readme.pdf}{Pi超算平台用户手册}



\renewcommand{\indexname}{索引}
\printindex
\end{document}